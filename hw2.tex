\documentclass{article}
\usepackage{amsmath,amsfonts,amsthm,fancyhdr,parskip,amssymb,graphicx}
\usepackage[all]{xy}
\usepackage[margin = 1.5in]{geometry}
\pagestyle{fancy}
\lhead{Ben Carriel (bkc39)}
\chead{Bryan Cuccioli (blc72)}
\rhead{Andy Levine (awl58)}
\setlength{\parindent}{1cm}

\DeclareMathOperator{\Z}{\mathbb{Z}}
\DeclareMathOperator{\Q}{\mathbb{Q}}
\DeclareMathOperator{\R}{\mathbb{R}}

\DeclareMathOperator{\oh}{\mathcal{O}}
\DeclareMathOperator{\ta}{\xrightarrow{\ \ \ }}
\DeclareMathOperator{\opt}{\texttt{OPT}}

\newcommand{\problem}[1]{\noindent {\bf #1}}
\newcommand{\problempart}[1]{\noindent{\textbf{(#1)}}}

\newtheorem*{thm}{Theorem}
\newtheorem*{lem}{Lemma}
\newtheorem*{claim}{Claim}
\newtheorem*{defn}{Definition}
\newtheorem*{prop}{Proposition}

\begin{document}

\problem{Problem 1.}

\problempart{a.} 

\problempart{b.}

\problempart{c.}

\problempart{d.}

\problem{Problem 2.}

We will construct a flow network $G$ with every edge having capacity 0 or 1. Assign capacity 1 to all pairs of nodes linked by edges in $G$ and 0 to the other pairs. Then each source in $A\cup B$ can only produce a flow of 1; we specify this by creating a super source $s^{\ast}$ with edges of capacity 1 into each source of $A\cup B$.

Now, augment the network by all augmenting paths starting from nodes in $A$ which give a flow $|A|$. This flow is suboptimal since there is possible flow at least $|B|>|A|$. Let $G_f$ be the residual graph induced by this flow. There is an edge-disjoint path to every node in $A$ from $t$ now that we have augmented these paths.

Moreover, there must now exist some augmenting path $p$, since our flow is sub-optimal. But each node in $A$ has only 1 unit of flow to push, so $p$ must originate from a node in $B\setminus A$. Denote this node $s$. Augmenting by $p$ gives an edge disjoint path from $t$ to every node in $A\cup\{s\}$ in $G_f$, and it follows that in $G$ there are edge-disjoint paths to $t$ for every $n\in A\cup\{s\}$.

\problem{Problem 3.}

\end{document}
